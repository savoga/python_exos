{\fontsize{12pt}{22pt} \textbf{Logistic regression}\par}

\vspace{5mm}

Logistic regression is used for binary classification.

It is quite similar to a simple linear regression in the sense that the objective is to find optimal weights $\omega$ to predict a variable. However, in the logistic regression we use a sigmoïd function.\\

Rem: "logistic" because the logistic law has a sigmoïd function as a repartition function.\\

\underline{Rationale behind the use of the sigmoïd function}:

We look for the \textit{à posteriori} probability $\mathbb{P}(y=1 | x) = \pi (x) = \hat{y}$.

The predicted variable $\hat{y}$ is thus a probability.  \\

The sigmoïd function: $\sigma: z \to \frac{1}{1+e^{-z}}$ is well adapted because of two reasons:

1) We want an output variable that is included in $[0,1]$ \\
2) $\frac{\pi(z)}{1-\pi(z)}$ represents the relationship between a distribution and its complementary (good in binary case), and it is just a transformation of $\sigma(z)=\frac{1}{1+e^{-z}}=\frac{e^z}{1-e^z}$ \\

Thus, we have: \\
$\hat{y} = \mathbb{P}(y=1 | x) = \sigma(\omega ^Tx + b) = \frac{1}{1-e^{-(\omega ^Tx + b)}}$

\vspace{5mm}

\underline{Estimation} \\
Estimation is done using maximum likelihood. Maximum likelihood is finding the parameter that maximizes the probability to have a specific event $(x_i, y_i)$ but in our case, it is a \textit{conditional} maximum likelihood since we want to maximize the \textit{à posteriori} probability that depends on $x$. \\

$L(\omega, b) = \prod_{i=1}^n \pi(x_i)^{y_i}(1-\pi(x_i))^{1-y_i}$ \\

This equation has no analytic solution. We use a numeric method to find the optimal parameters (see optimizaton algorithms).

See \textit{Neural Network} section for more details on optimization.

\vspace{5mm}