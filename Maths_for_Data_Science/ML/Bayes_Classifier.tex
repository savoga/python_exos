{\fontsize{12pt}{22pt} \textbf{Bayes classifier}\par}

\vspace{5mm}

$g$ is the \textit{classifier}.

$$g: \mathcal{X} \to \mathcal{Y}$$
$$~~~~~~~~~~\mathbb{R}^d \to \{0,1\}$$

To model the learning problem, we use the pair $(X,Y)$ described by $(\mu, \eta)$ where $\mu$ is the probability measure:

$$\mu(A) = \mathbb{P}(X \in A)$$

And $\eta$ is the regression of $Y$ on $X$:

$$\eta(X) = \mathbb{P}(Y=1 | X=x) = \mathbb{E}[Y | X=x]$$

$\eta$ is also called the \textit{a posteriori probability}.

The Bayes classifier is:

  \begin{equation}
    \begin{cases}
      1 & \text{if}\ \eta(x) > 1/2 \\
      0 & \text{otherwise}
    \end{cases}
  \end{equation}

Or, if $\mathcal{Y}$ is $\{-1,1\}$, we write the classifier as such: $g(x) = 2 \mathbbm{1}\{ \eta(x)>1/2\}-1$.

\vspace{5mm}

\underline{Theorem}:

\vspace{5mm}

For any classifier g: $\mathbb{R}^d \to \{0,1\}$,
$$\mathbb{P}(g^*(X) \neq Y) \le \mathbb{P}(g(X) \neq Y)$$

In other words, the Bayes classifier is theorically \textbf{the best classifier}.

\vspace{5mm}

\textit{Proof}: express $\mathbb{P}(g(X) \neq Y) - \mathbb{P}(g^*(X) \neq Y)$ in terms of dummies (use complementaries) and show that it is superior to 0.

\vspace{5mm}