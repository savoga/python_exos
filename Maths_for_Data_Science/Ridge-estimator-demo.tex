% documentclass
% set font size=11 (11pt)
% set paper format=A4 (a4paper)
% set equation alignment to left (fleqn)
\documentclass[11pt,a4paper,fleqn]{article}


% Preamble
% use the inputenc and fontenc packages to use French accents
\usepackage[utf8]{inputenc}
\usepackage[T1]{fontenc}
% allow for arbitrary font size
\usepackage{anyfontsize}
% set the font as Time New Roman (the Latex equivalent, at least)
% \usepackage{mathptmx}
% set the size of the document margins using the geometry package
\usepackage[lmargin=0.97in,rmargin=0.97in,tmargin=1.4in,bmargin=1.4in]{geometry}
% turn the color of footnote markers to black
\renewcommand\thefootnote{\textcolor{black}{\arabic{footnote}}}
% suppress indents on footnotes
\usepackage[hang,flushmargin]{footmisc}
% automatically generates colored brackets around references
\usepackage{fncylab} \labelformat{equation}{(#1)}
% supress indent on new paragraphs
\setlength{\parindent}{0pt}
% use the amsmath package to include mathematical symbols
\usepackage{amsmath}
% suppress the space between the left margin and the equations (fleqn still leaves some space by default)
\setlength{\mathindent}{0pt}
% create a new environment to left flush the equation with the align environment
\makeatletter
\newenvironment{lflalign}{ \vspace{-3mm}%
  \def\align@preamble{%
    &\strut@
    \setboxz@h{\@lign$\m@th\displaystyle{####}$}%
    \ifmeasuring@\savefieldlength@\fi
    \set@field
    \hfil
    \tabskip\z@skip
    &\setboxz@h{\@lign$\m@th\displaystyle{{}####}$}%
    \ifmeasuring@\savefieldlength@\fi
    \set@field
    \hfil
    \tabskip\alignsep@
  }
  \flalign}
{\endflalign}
\makeatother
% use the ammssymb package to use mathematical symbols
\usepackage{amssymb}
% create new commands for mathematical symbols
\DeclareMathOperator{\N}{\mathbb{N}}
\DeclareMathOperator{\Z}{\mathbb{Z}}
\DeclareMathOperator{\Q}{\mathbb{Q}}
\DeclareMathOperator{\R}{\mathbb{R}}
\DeclareMathOperator{\Pb}{\mathbb{P}}
% declare the cmsy (computer modern symbol) math alphabet to define appropriate fonts for the U and N mathematical symbols
\DeclareMathAlphabet\mathbcal{OMS}{cmsy}{m}{n}
% create new commands for mathematical symbols
\DeclareMathOperator{\E}{\mathbcal{E}}
\DeclareMathOperator{\Ex}{\mathbb{E}}
\DeclareMathOperator{\F}{\mathbcal{F}}
\DeclareMathOperator{\G}{\mathbcal{G}}
\DeclareMathOperator{\M}{\mathbcal{M}}
\DeclareMathOperator{\HH}{\mathbcal{H}}
\DeclareMathOperator{\QQ}{\mathbcal{Q}}
\DeclareMathOperator{\PP}{\mathbcal{P}}
\DeclareMathOperator{\Noo}{\mathbcal{N}}
\DeclareMathOperator{\U}{\mathbcal{U}}
% use the bbm package to be able to use the double stroke 1 for the indicator function
\usepackage{bbm}
\DeclareMathOperator{\ind}{\mathbbmss{1}}
% use the bm package to use bold characters in math mode
\usepackage{bm}
% create a new command for black square bullets
\newcommand{\bs}{\scalebox{0.7}{$\blacksquare$} \hspace{2mm}}
% use the relsize package to be abe to change the size of mathematical symbols
\usepackage{relsize}
% define a new command for in-line small summation
\newcommand{\ssumm}[2]{\underset{\scriptscriptstyle #1}{\overset{\scriptscriptstyle #2}{\mathlarger{\mathlarger{\mathlarger{\Sigma}}}}} \hspace{0.5mm}}
% define a new command for in-line small products
\newcommand{\sprod}[2]{\underset{\scriptscriptstyle #1}{\overset{\scriptscriptstyle #2}{\mathlarger{\mathlarger{\mathlarger{\Pi}}}}} \hspace{0.5mm}}
% Use the caption package to customize captions (titles) of tables and graphs
\usepackage[font=small,labelfont=bf]{caption}
% use float package to force figure the be positioned where indicated
\usepackage{float}
% use the graphicx package to be able to resize tables
\usepackage{graphicx}


\begin{document}

On note $\hat \theta = {\rm arg} \displaystyle\min_{\theta} \frac{1}{2}\| \mathbf{y} -X \theta \|_2^2+ \frac{\lambda}{2}\|\theta\|_2^2$ l’estimateur Ridge

Soit $f: \theta \mapsto \frac{1}{2}||Y-X\theta||_2^2+\frac{\lambda}{2}||\theta||_2^2$ \\

1) Quand $X=Id_n$ on a $n=p+1$ et $f(\theta)=\frac{1}{2}||Y-Id_n\theta||_2^2+\frac{\lambda}{2}||\theta||_2^2$ \\

Pour des questions de notations, on utilisera $Id_n=Id_{p+1}=Id$ \\

$f(\theta)=\frac{1}{2}\Sigma_{i=1}^n(Y_i-Id_i\theta)^2+\frac{\lambda}{2} \Sigma_{i=1}^n \theta_i^2$ \\

Le minimum de la fonction est atteint lorsque $\nabla f(\theta) = 0$ \\

$\nabla f(\theta) = 0 => \forall k=1,...,p: \frac{\partial f(\theta)}{\partial \theta_k}=0$ \\

$\frac{\partial f(\theta)}{\partial \theta_k}=2\frac{1}{2} (-Id_{i,k}) \Sigma_{i=1}^n (Y_i-\Sigma_{j=1}^pId_{i,j}\theta_j)+2\frac{\lambda}{2} \theta_k$ \\

$~~~~~~~=-\Sigma_{i=1}^n(Id_{i,k}Y_i-Id_{i,k}Id_i\theta)+\lambda \theta_k$ \\

Donc $\nabla f(\theta)=-\Sigma_{i=1}^n(Id_i^TY_i-Id_i^TId_i\theta)+\lambda \theta=-\Sigma_{i=1}^nId_i^TY_i+\Sigma_{i=1}^nId_i^TId_i \theta +\lambda \theta$ \\

$~~~~~~~~~~~~~~~~~=-Y+\theta+\lambda \theta$ \\

$\nabla f(\theta) = 0 => \theta = \frac{1}{1+\lambda}Y= \hat{\theta}_{n,\lambda}^{Ridge} $\\


2) Pour $X \in \mathbb{R}^{n \times p}$ quelconque, $f(\theta)=\frac{1}{2}\Sigma_{i=1}^n(Y_i-X_i\theta)^2+\frac{\lambda}{2} \Sigma_{i=1}^n \theta_i^2$ \\

Comme pour la question 1), \\
$\nabla f(\theta) = 0 => \forall k=1,...,p: \frac{\partial f(\theta)}{\partial \theta_k}=0$ \\

$\frac{\partial f(\theta)}{\partial \theta_k}=2\frac{1}{2} (-X_{i,k}) \Sigma_{i=1}^n (Y_i-\Sigma_{j=1}^pX_{i,j}\theta_j)+2\frac{\lambda}{2} \theta_k$ \\

$~~~~~~~=-\Sigma_{i=1}^n(X_{i,k}Y_i-X_{i,k}X_i\theta)+\lambda \theta_k$ \\

Donc $\nabla f(\theta)=-\Sigma_{i=1}^n(X_i^TY_i-X_i^TX_i\theta)+\lambda \theta=-X^TY+X^TX\theta + \lambda \theta$ \\

Ainsi $\nabla f(\theta)=0 => -X^TY+X^TX\theta + \lambda \theta=0$ \\
$=> (X^TX + \lambda Id_n)\theta = X^TY => \theta = (X^TX + \lambda Id_n)^{-1} X^TY = \hat{\theta}_{n,\lambda}^{Ridge} $ \\


\end{document} 
