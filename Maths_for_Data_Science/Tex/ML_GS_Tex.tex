% documentclass
% set font size=11 (11pt)
% set paper format=A4 (a4paper)
% set equation alignment to left (fleqn)
\documentclass[11pt,a4paper,fleqn]{article}


% Preamble
% use the inputenc and fontenc packages to use French accents
\usepackage[utf8]{inputenc}
\usepackage[T1]{fontenc}
% allow for arbitrary font size
\usepackage{anyfontsize}
% set the font as Time New Roman (the Latex equivalent, at least)
% \usepackage{mathptmx}
% set the size of the document margins using the geometry package
\usepackage[lmargin=0.97in,rmargin=0.97in,tmargin=1.4in,bmargin=1.4in]{geometry}
% turn the color of footnote markers to black
\renewcommand\thefootnote{\textcolor{black}{\arabic{footnote}}}
% suppress indents on footnotes
\usepackage[hang,flushmargin]{footmisc}
% automatically generates colored brackets around references
\usepackage{fncylab} \labelformat{equation}{(#1)}
% supress indent on new paragraphs
\setlength{\parindent}{0pt}
% use the amsmath package to include mathematical symbols
\usepackage{amsmath}
% suppress the space between the left margin and the equations (fleqn still leaves some space by default)
\setlength{\mathindent}{0pt}
% create a new environment to left flush the equation with the align environment
\makeatletter
\newenvironment{lflalign}{ \vspace{-3mm}%
  \def\align@preamble{%
    &\strut@
    \setboxz@h{\@lign$\m@th\displaystyle{####}$}%
    \ifmeasuring@\savefieldlength@\fi
    \set@field
    \hfil
    \tabskip\z@skip
    &\setboxz@h{\@lign$\m@th\displaystyle{{}####}$}%
    \ifmeasuring@\savefieldlength@\fi
    \set@field
    \hfil
    \tabskip\alignsep@
  }
  \flalign}
{\endflalign}
\makeatother
% use the ammssymb package to use mathematical symbols
\usepackage{amssymb}
% create new commands for mathematical symbols
\DeclareMathOperator{\N}{\mathbb{N}}
\DeclareMathOperator{\Z}{\mathbb{Z}}
\DeclareMathOperator{\Q}{\mathbb{Q}}
\DeclareMathOperator{\R}{\mathbb{R}}
\DeclareMathOperator{\Pb}{\mathbb{P}}
% declare the cmsy (computer modern symbol) math alphabet to define appropriate fonts for the U and N mathematical symbols
\DeclareMathAlphabet\mathbcal{OMS}{cmsy}{m}{n}
% create new commands for mathematical symbols
\DeclareMathOperator{\E}{\mathbcal{E}}
\DeclareMathOperator{\Ex}{\mathbb{E}}
\DeclareMathOperator{\F}{\mathbcal{F}}
\DeclareMathOperator{\G}{\mathbcal{G}}
\DeclareMathOperator{\M}{\mathbcal{M}}
\DeclareMathOperator{\HH}{\mathbcal{H}}
\DeclareMathOperator{\QQ}{\mathbcal{Q}}
\DeclareMathOperator{\PP}{\mathbcal{P}}
\DeclareMathOperator{\Noo}{\mathbcal{N}}
\DeclareMathOperator{\U}{\mathbcal{U}}
% use the bbm package to be able to use the double stroke 1 for the indicator function
\usepackage{bbm}
\DeclareMathOperator{\ind}{\mathbbmss{1}}
% use the bm package to use bold characters in math mode
\usepackage{bm}
% create a new command for black square bullets
\newcommand{\bs}{\scalebox{0.7}{$\blacksquare$} \hspace{2mm}}
% use the relsize package to be abe to change the size of mathematical symbols
\usepackage{relsize}
% define a new command for in-line small summation
\newcommand{\ssumm}[2]{\underset{\scriptscriptstyle #1}{\overset{\scriptscriptstyle #2}{\mathlarger{\mathlarger{\mathlarger{\Sigma}}}}} \hspace{0.5mm}}
% define a new command for in-line small products
\newcommand{\sprod}[2]{\underset{\scriptscriptstyle #1}{\overset{\scriptscriptstyle #2}{\mathlarger{\mathlarger{\mathlarger{\Pi}}}}} \hspace{0.5mm}}
% Use the caption package to customize captions (titles) of tables and graphs
\usepackage[font=small,labelfont=bf]{caption}
% use float package to force figure the be positioned where indicated
\usepackage{float}
% use the graphicx package to be able to resize tables
\usepackage{graphicx}


\begin{document}

% command to check unused bibliography entries
% \nocite{*}
{\fontsize{12pt}{22pt} \textbf{Logistic regression}\par}

\vspace{5mm}

Logistic regression is used for binary classification.

It is quite similar to a simple linear regression in the sense that the objective is to find optimal weights $\omega$ to predict a variable. However, in the logistic regression we use a sigmoïd function.\\

Rem: "logistic" because the logistic law has a sigmoïd function as a repartition function.\\

\underline{Rationale behind the use of the sigmoïd function}:

We look for the \textit{à posteriori} probability $\mathbb{P}(y=1 | x) = \pi (x) = \hat{y}$.

The predicted variable $\hat{y}$ is thus a probability.  \\

The sigmoïd function: $\sigma: z \to \frac{1}{1+e^{-z}}$ is well adapted because of two reasons:

1) We want an output variable that is included in $[0,1]$ \\
2) $\frac{\pi(z)}{1-\pi(z)}$ represents the relationship between a distribution and its complementary (good in binary case), and it is just a transformation of $\sigma(z)=\frac{1}{1+e^{-z}}=\frac{e^z}{1-e^z}$ \\

Thus, we have: \\
$\hat{y} = \mathbb{P}(y=1 | x) = \sigma(\omega ^Tx + b) = \frac{1}{1-e^{-(\omega ^Tx + b)}}$

\vspace{5mm}

\underline{Estimation} \\
Estimation is done using maximum likelihood. Maximum likelihood is finding the parameter that maximizes the probability to have a specific event $(x_i, y_i)$ but in our case, it is a \textit{conditional} maximum likelihood since we want to maximize the \textit{à posteriori} probability that depends on $x$. \\

$L(\omega, b) = \prod_{i=1}^n \pi(x_i)^{y_i}(1-\pi(x_i))^{1-y_i}$ \\

This equation has no analytic solution. We use a numeric method to find the optimal parameters (see optimizaton algorithms).

See \textit{Neural Network} section for more details on optimization.

\vspace{5mm}
\end{document} 
