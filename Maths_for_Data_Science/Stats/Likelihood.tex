{\fontsize{12pt}{22pt} \textbf{Likelihood method}\par}

\vspace{5mm}

This method consists on finding the parameter that maximizes the likelihood:

$L(x_1,...,x_n;\theta) = f(X | \theta) = \prod_{i=1}^{n}f_{\theta}(x_i;\theta)$ which is the product of densities across all samples.

\vspace{5mm}

Intuitively, we want to find the $\theta$ that maximizes a certain event, that is, obtaining some data $X$ (which is why we have $X | \theta$).

We often use the log in order to get rid of power coefficients appearing with the product. \\
\textit{likelihood equation}: $\frac{d}{d\theta}ln(L(x_1,...,x_n;\theta))=0$

\vspace{5mm}

\textit{Note}: in machine learning, we use likelihood maximization in unsupervised learning when we want to estimate parameters of a distribution sample (generative models).

\vspace{5mm}