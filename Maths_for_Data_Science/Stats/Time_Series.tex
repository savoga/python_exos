{\fontsize{12pt}{22pt} \textbf{Time series}\par}

\vspace{5mm}

\underline{Differential equations}

\vspace{5mm}

A differential equation is an equation with the following characteristics:

- variables = functions

- it expresses the relationship of functions (variables) with their derivatives

\vspace{5mm}

Case of \textit{linear and constant coefficient} differential equations:

\vspace{5mm}

\begin{center}
$a_ny^{(n)} + a_{n-1}y^{(n-1)} + ... + a_1y' + a_0y = 0$
\end{center}

$(n)$: n-th derivative

\vspace{5mm}

In order to solve such equations, we use \textit{characteristic equations}:

$a_ny^{(n)} + a_{n-1}y^{(n-1)} + ... + a_1y' + a_0y = 0$ (E)

Let $y(x) = e^{rx}$

(E) => $a_nr^n e^{rx} + a_{n-1}r^{(n-1)} e^{rx} + ... + a_1 r e^{rx} + a_0e^{rx} = 0$

Since $e^{rx} \neq 0$

(E) => $a_nr^n + a_{n-1}r^{(n-1)} + ... + a_1 r + a_0 = 0$

We thus end up with a polynomial function.

In order to find the general solution of (E), we can find the solution of the characteristic equation and deduce the general solution (using exponential).

\vspace{5mm}

\underline{Autoregressive processes}

\vspace{5mm}

Autoregressive processes are a specifc case of \textit{differential equations}.

\vspace{5mm}

$y_{t+k} = \beta_1 y_{t+k-1} + \beta_2 y_{t+k-2} + ... + \beta_k y_{t}$

\vspace{5mm}

Characteristic equation:

$r^k - \beta_1 r^{k-1} - ... - \beta_{k-1} r - \beta_k = 0$

\vspace{5mm}

\underline{Stationary processes}

\vspace{5mm}

A stationary process has the same moment (expectation, variance, etc.)  in every single point.

In practice, we check the stationarity with only the first two moments (expectation and variance).

\vspace{5mm}

Intuition behind the importance of stationary processes in regressions:

When performing regressions, it is important to make sure the error term is stationary. If non stationary, there's probably a trend that is not caught by the explanatory variables used. This can lead to \textit{spurious regressions}.

\vspace{5mm}

To make sure a process is stationary, we have to check the existence of a \textit{unit root}.

\vspace{5mm}

Why existence of unit root leads to non-stationary process?

Toy example:

Let us consider a 1st order autoregressive process $y_t = \beta_0 + \beta_1 y_{t-1} + \epsilon_t$

Let $\beta_0 = 0$. The characteristic equation is:

$r - \beta_1 = 0$

The solution is $r = \beta_1$

The problem has thus a unit root when $\beta_1 = 1$

Since $y_t = \beta_0 + \beta_1 y_{t-1} + \epsilon_t$ we can write:

$y_1 = y_0 + \epsilon_0$

$y_2 = y_1 + \epsilon_1 = y_0 + \epsilon_0 + \epsilon_1$

$y_3 = y_0 + \epsilon_0 + \epsilon_1 + \epsilon_2$

Thus, $y_t = y_0 + \Sigma_{j=0}^t \epsilon_j$

The variance is $\mathbb{V}[y_t] = t \sigma^2$ (we assume a constant variance for $\epsilon$)

Consequently, the variance is increasing with time so the process is \textbf{not stationary}.

\vspace{5mm}

To detect stationarity, we can perform a unit root test such as \textit{Augmented Dicky Fuller test}.

\vspace{5mm}

Non stationarity can be corrected in several ways :

- time regression : performing a regression on time and working with the error term

Example : if $y_t$ in non stationary

$y_t = \beta_0 + \beta_1 t +\epsilon_t$ --> $\epsilon_t$ will not depend on time anymore

- finite differences : removing previous term to each observation $y_t = y_t - y_{t-1}$ --> this will have the effect to remove the trend

- moving average XxX

Example : using (double) centered moving average 5x5 

\lstset{language=Python}
\lstset{frame=lines}
\lstset{caption={Centered moving average (double)}}
\lstset{label={lst:code_direct}}
\lstset{basicstyle=\footnotesize}
\begin{lstlisting}

cpi_roll = cpi.rolling(window=5).mean() # cell at index 4 is the mean of the 5 previous ones (inclusive)
cpi_mm = cpi - cpi_roll
cpi_roll_2 = cpi_mm.rolling(window=5).mean()
cpi_mm_2 = cpi_mm - cpi_roll_2

\end{lstlisting}

\vspace{5mm}